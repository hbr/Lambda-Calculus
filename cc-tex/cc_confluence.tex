\section{Confluence}
\label{sec:Confluence}
%------------------------------------------------------------

\begin{comment}

    Why: Make lambda calculus really a model of computation. Result (i.e. a
    term in normal form) is independent of the reduction sequence.

    Goal: Uniqueness of normal forms and beta equivalent terms have a common
    reduct. Corollaries: Equivalent binders have equivalent subterms and binders
    cannot reduce to be beta equivalent with sorts or variables.

    Basic Property: Beta reduction is confluent.


    Prove of confluence:

    - Definition via diamonds

    - Diamond between a relation and its reflexive transitive closure proves the
    confluence of the relation.

    - Parallel beta reduction is a diamond between beta reduction and its
    reflexive transitive closure.


\end{comment}




\subsection{Overview}

We want to be able to use the calculus of constructions for computation.
The basic computation step is beta reduction. A computation in the calculus ends
if no more reduction is possible i.e. if the term has been reduced to a normal
form.

However beta reduction is ambiguous. There might be more than one possibility to
make a beta reduction.

\begin{example}
    \label{ex:Confluence}
    $$
    \begin{array}{cccccc}
        & & (\lambda x. f x x) ((\lambda y.y) z)
        \\
        & \swarrow_{\beta}
        & & \searrow_\beta
        \\
        f ((\lambda y.y) z) ((\lambda y.y) z)
        \\
        \downarrow_\beta
        & & & & (\lambda x . f x x) z
        \\
        f z ((\lambda y.y) z)
        \\
        & \searrow_\beta
        & &  \swarrow_\beta
        \\
        & & f z z
    \end{array}
    $$
\end{example}

In the left path we reduce first the outer redex. This reduction duplicates the
redex in the argument. Therefore we need three reduction steps to reach the
final term $f z z$.

In the right path we reduce first the redex in the argument of the application
and then the outer redex. Therefore we need only two reduction steps to reach
the final term $f z z$.

In that specific example we have shown that both paths end up in the same
result. However we have to prove that this is always the case. Otherwise the
calculus of constructions would be useless as a calculus.

It turns out that \emph{confluence} is a key property in the proof of uniqueness
of results. We define a relation as \emph{confluent} if going an arbitrary
number of steps in two directions there are always continuations of the paths to
join them.  We want to prove confluence of beta reduction.

Before proving confluence of beta reduction we define the confluence of a
relation by defining first a diamond property of a relation. The diamond
property of a relation is something like a \emph{one step confluence}. Then we
show that the confluence of a relation can be proved by finding a diamond
between it and its reflexive transitive closure.

In a next step we define a parallel beta reduction, prove that it is a diamond
between beta reduction and the reflexive transitive closure of beta reduction.
This proves the confluence of beta reduction.

Having the confluence of beta reduction it is easy to prove that beta equivalent
terms always have a common reduct, normal forms are unique and some other
interesting properties of binders.





\subsection{Diamonds and Confluence}


\begin{definition}
    \emph{Diamond Property} A relation $r$ has the diamond property (or is a
    diamond) if $a \torel{r} b$ and $a \torel{r} c$ implies the existence of
    some $d$ such that $b \torel{r} d$ and $c \torel{r} d$ are valid.

    $$
    \ruleh{
        a \torel r b
        \\
        a \torel r c
    }
    {
        \exists d. b \torel{r} d \land c \torel{r} d
    }
    $$
\end{definition}

We can state the diamond property of a relation $r$ more pictorially as
$$
\begin{array}{lll}
    a   &  \torel{r} &   b
    \\
    \downarrow_r & & \downarrow_r
    \\
    c & \torel{r} & \exists d
\end{array}
$$
which looks like a diamond if tilted by $45^\circ$ clockwise.


\begin{definition}
    \emph{Confluence} A relation $r$ is confluent if its reflexive transitive
    closure $r^*$ is a diamond.
\end{definition}


\begin{theorem}
    \label{RelationBetween}
    \emph{Let $r$ and $s$ be two relations. If $s$ is between $r$ and its
    reflexive transitive closure i.e. $r \subseteq s \subseteq r^*$ then both
    closures are the same i.e. $r^* = s^*$.}

    \begin{proof}
        Note that $r \subseteq s$ is defined by $\ruleall{ab}{a \torel{r} b}{a
        \torel{s} b}$.

        Assume $r \subseteq s \subseteq r^*$.
        %
        In order to prove $r^* = s^*$ we have to prove $r^* \subseteq s^*$ and
        $s^* \subseteq r^*$.

        \begin{enumerate}
            \item $r^* \subseteq s^*$: We have to prove the goal
                $$
                \ruleh{a \torel{r^*} b}{a \torel{s^*} b}
                $$
                We use induction on $a \torel{r^*} b$.
                \begin{enumerate}
                    \item $a = b$: Trivial since $a \torel{s*} a$ is valid by
                        definition.

                    \item
                        $$
                        \begin{array}{l|l}
                            a \torel{r^*} b
                            &
                            a \torel{s^*} b
                            \\
                            b \torel{r} c
                            \\
                            \hline
                            a \torel{r^*} c
                            &
                            a \torel{s^*} c
                        \end{array}
                        $$
                        In order to prove the final goal in the lower right
                        corner we start from the induction hypothesis $a
                        \torel{s^*} b$.

                        Since $r \subseteq s$ we have $b \torel{s} c$ by
                        definition of $\subseteq$ and the second premise.

                        Then by definition of the reflexive transitive closure
                        we conclude the final goal.
                \end{enumerate}

            \item $s^* \subseteq r^*$: We have to prove the goal
                $$
                \ruleh{a \torel{s^*} b}{a \torel{r^*} b}
                $$
                We use induction on $a \torel{s^*} b$.
                \begin{enumerate}
                    \item $a = b$: Trivial since $a \torel{r*} a$ is valid by
                        definition.

                    \item
                        $$
                        \begin{array}{l|l}
                            a \torel{s^*} b
                            &
                            a \torel{r^*} b
                            \\
                            b \torel{s} c
                            \\
                            \hline
                            a \torel{s^*} c
                            &
                            a \torel{r^*} c
                        \end{array}
                        $$
                \end{enumerate}

                Start with the induction hypothesis $a \torel{r^*} b$.

                Since $s \subseteq r^*$ is given we can infer $b \torel{r^*} c$
                from the second premise.

                $r^*$ is transitive and therefore $a \torel{r^*} c$ must be
                valid.
        \end{enumerate}
    \end{proof}
\end{theorem}




\begin{lemma}
    \label{StripLemma1}
    If $r$ is a diamond then
    $$
    \diamond a b c d {r^*} r
    $$
    is valid.

    \begin{proof}
        By induction on $a \torel{r^*} b$.

        \begin{enumerate}
        \item In the reflexive case we have $a = b$. We choose $d = c$ which
        satisfies the required properties.

        \item
            $$
            \begin{array}{l|l}
                a \torel{r^*} b
                &
                \allbracket d {\diamond a b d e {r^*} r}
                \\
                b \torel r c
                \\
                \hline
                a \torel{r^*} c
                &
                \allbracket d {\diamond a c d f {r^*} r}
            \end{array}
            $$
            To prove the goal in the lower right corner we assume $a \torel r d$
            and try to find some $f$ with the required properties.

            From the induction hypothesis we find some $e$ which satisfies $b
            \torel r e$ and $d \torel{r^*} e$.

            Since $r$ is a diamond there exists some $f$ with $\diamond b c e f
            r r$. By glueing the boxes together we see that $f$ satisfies
            the required properties.
            $$
            \begin{array}{ccccc}
                a            &\torel{r^*}& b           &\torel r& c
                \\
                \downarrow_r &           & \downarrow_r&        & \downarrow_r
                \\
                d            &\torel{r^*}& e           &\torel r& f
            \end{array}
            $$
        \end{enumerate}
    \end{proof}

\end{lemma}



\begin{lemma}
    \label{StripLemma2}
    \emph{ If $r$ is a diamond, then $r^*$ is a diamond as well.}
    $$
    \diamond a b c d {r^*} {r^*}
    $$

    \begin{proof}
        We assume that $r$ is a diamond and $a \torel{r^*} b$ and do induction
        on $a \torel{r^*} c$

        \begin{enumerate}
            \item In the reflexive case we have $a = c$. We use $d = b$ which
                satisfies the required properties.

            \item
                $$
                \begin{array}{l|l}
                    a \torel{r^*} c
                    &
                    \diamond a b c e {r^*} {r^*}
                    \\
                    c \torel r d
                    \\
                    \hline
                    a \torel{r^*} d
                    &
                    \diamond a b d f {r^*} {r^*}
                \end{array}
                $$
                We have to prove the goal in the lower right corner i.e. find
                some $f$ with the required properties.

                From the induction hypothesis we find some $e$. Using the
                previous lemma~\ref{StripLemma1} we find some $f$ such that we
                can glue the boxes together in order to see that $f$ satisfies
                the required properties.
                $$
                \begin{array}{ccc}
                    a & \torel{r^*} & b
                    \\
                    \downarrow_{r^*} & & \downarrow{r^*}
                    \\
                    c & \torel{r^*} & e
                    \\
                    \downarrow_r & & \downarrow_r
                    \\
                    d & \torel{r^*} & f
                \end{array}
                $$
        \end{enumerate}
    \end{proof}
\end{lemma}


Now we can prove the basic theorem of this subsection:

\begin{theorem}
    \label{DiamondBetween}
    \emph{In order to prove the confluence of a relation it is sufficient to
    find a diamond between it and its reflexive transitive closure}.

    $$
    \rulev{
        r \subseteq s \subseteq r^*
        \\
        s \text{ is a diamond}
    }
    {
        r \text{ is confluent}
    }
    $$

    \begin{proof}
        Assume $r \subseteq s \subseteq r^*$ and $s$ is a diamond.

        By~\ref{RelationBetween} we know that both reflexive transitive closures
        are the same i.e. $r^* = s^*$.

        From~\ref{StripLemma2} we conclude that $s^*$ is a diamond. This implies
        that $r^*$ is a diamond as well and proves the fact that $r$ is
        confluent by definition of confluence.
    \end{proof}
\end{theorem}










\subsection{Parallel Reduction Relation}

Looking at the example~\ref{ex:Confluence} it can be seen that beta reduction is
not a diamond. Let's analyze the example a little bit to see why beta reduction
is not a diamond. Assume there is a redex like
$$
    (\lambda x^A. e) a
$$
where the subterm $a$ contains redexes as well i.e. there is some $b$ with $a
\reduce b$. Then the following two reduction paths are possible.
$$
\begin{array}{lllll}
    (\lambda x^A. e) a
    &\reduce& (\lambda x^A. e) b
    &\reduce& e[x:=b]
    \\
    (\lambda x^A. e) a
    &\reduce& e[x:=a]
    &\reducestar& e[x:=b]
\end{array}
$$

In the first path we reach the term $e[x:=b]$ in two reduction steps. In the
second path we can reach the term $e[x:=b]$ as well. But the needed number of
steps depends on how often the variable $x$ is present in the term $e$. If $x$
is contained in the term $e$ $n$ times we need $n+1$ reduction steps in total
because we have to reduce the term $a$ in $e[x:=a]$ to the term $b$ $n$ times.
If beta reduction were a diamond, it would be required that $n$ is always $1$
which is not the case in general.

In order to prove the confluence of beta reduction we have to find a diamond
between beta reduction and its reflexive transitive closure. Let's call this
relation $\beta_p$. This relation is like beta reduction $\beta$ but must allow
more reductions steps to remedy the situation that in a redex $(\lambda x^A.e)a$
the variable $x$ might be contained zero or more times in the subterm $e$.

\begin{itemize}
    \item In order to remedy the situation that the variable $x$ is not
        contained in the subterm $e$ we make the relation $\beta_p$ reflexive.

    \item In order to remedy the situation that the variable $x$ is contained
        two or more times in the subterm $e$ we allow that $\beta_p$ reductions
        can happen in any subterm of a term. Therefore we call $\beta_p$
        \emph{parallel} beta reduction because the reduction can happen in the
        subterms in parallel.
\end{itemize}


\begin{definition}
    The \emph{parallel reduction relation} $a \preduce b$ is defined inductively
    by the rules
    \begin{enumerate}
    \item Reflexive
    $$
        a \preduce a
    $$

    \item Redex
    $$
        \rulev {
            a \preduce b
            \\
            e \preduce f
        }
        {
            (\lambda x^A. e) a \preduce f[x:=b]
        }
    $$

    \item Product
    $$
    \rulev {
        A \preduce C
        \\
        B \preduce D
    }
    {
        \Pi x^A. B \preduce \Pi x^C. D
    }
    $$

    \item Abstraction
    $$
    \rulev {
        A \preduce B
        \\
        e \preduce f
    }
    {
        \lambda x^A. e \preduce \lambda x^B. f
    }
    $$

    \item Application
    $$
    \rulev {
        a \preduce c
        \\
        b \preduce d
    }
    {
        a b \preduce c d
    }
    $$
    \end{enumerate}
\end{definition}


\begin{lemma}
    \label{ReductionLeParallelReduction}
    \emph{Parallel beta reduction is a superset of beta reduction}

    \begin{proof} This fact is trivial, because all rules of beta reduction are
    contained as special cases within the rules of parallel beta reduction.
    \end{proof}
\end{lemma}



\begin{lemma}
    \label{ParallelReductionLeReductionStar}
    \emph{Parallel beta reduction is a subset of the transitive closure of beta
    reduction}

    \begin{proof}
        All rules of parallel beta reduction are satisfied by the transitive
        closure of beta reduction. Since an inductive relation defined by some
        rules is the smallest relation which satisfies the rules, it is evident
        that parallel beta reduction must be smaller than the transitive closure
        of beta reduction.
    \end{proof}
\end{lemma}







\begin{lemma}
    \label{ParallelReductionSubstitution1}
    \emph{Basic compatibility of parallel reduction and substitution}
    $$
    \ruleh{
        t \preduce u
    }
    {
        a[x:=t] \preduce a[x:=u]
    }
    $$

    \begin{proof}
    By induction on the structure of $a$.
    \begin{enumerate}
    \item $a$ is a sort: Trivial, because substitution does not change a sort
    and parallel beta reduction is reflexive.

    \item $a$ is a variable, let's say $y$: In the case $x=y$ the goal is
    implied by the premise. In the case $x \ne y$ the goal is implied by
    reflexivity.

    \item $a$ is the product $\Pi y^B. C$: We have to prove the goal
    $(\Pi y^B. C)[x:=t] \preduce (\Pi y^B.C)[x:=u]$ from the premise $t \preduce
    u$ and the induction hypotheses  $B[x:=t] \preduce B[x:=u]$ and
    $C[x:=t] \preduce C[x:=u]$. The validity of the goal can be seen from the
    following derivation.
    $$
    \begin{array}{lllll}
        (\Pi y^B. C)[x:=t]
        &=& \Pi y^{B[x:=t]}. C[x:=t]
        &\text{definition of substitution}
        \\
        &\preduce& \Pi y^{B[x:=u]}. C[x:=u]
        &\text {induction hypothesis}
        \\
        &=& (\Pi y^B. C)[x:=u]
        &\text{definition of substitution}
    \end{array}
    $$

    \item $a$ is the abstraction $\lambda y^B. e$: Same reasoning as with
    product.

    \item $a$ is the application $f a$: Same reasoning as with
    product.
    \end{enumerate}
    \end{proof}
\end{lemma}


\begin{lemma}
    \label{ParallelReductionSubstitution2}
    \emph{Full compatibility of parallel reduction and substitution}
    $$
    \rulev{
        a \preduce b
        \\
        t \preduce u
    }
    {
        a[x:=t] \preduce b[x:=u]
    }
    $$

    \begin{proof}
        By induction on $a \preduce b$.

        \begin{enumerate}
        \item Reflexive: In that case we have $a = b$. The goal is an immediate
        consequence of lemma~\ref{ParallelReductionSubstitution1}

        \item Redex:
        $$
        \begin{array}{l|l}
            a \preduce b
            & a[x:=t] \preduce b[x:=u]
            \\
            e \preduce f
            & e[x:=t] \preduce f[x:=u]
            \\
            \hline
            (\lambda y^A. e) a \preduce f[y:=b]
            &
            ((\lambda y^A. e) a)[x:=t] \preduce f[y:=b][x:=u]
        \end{array}
        $$

        The validity of the goal in the lower right corner can be seen from the
        following reasoning:
        $$
        \begin{array}{lllll}
            ((\lambda y^A. e) a)[x:=t]
            &=& (\lambda y^{A[x:=t]}. e[x:=t]) a[x:=t]
            &\text{definition of substitution}
            \\
            &\preduce& f[x:=u][y:=b[x:=u]]
            &\text{induction hypotheses}
            \\
            &=& f[y:=b][x:=u]
            &\text{double substitution~\ref{DoubleSubstitution}}
        \end{array}
        $$

        \item Product, abstraction and application: Some reasoning as with
        \emph{redex}. The lemma~\ref{DoubleSubstitution} is not needed in these
        cases.
        \end{enumerate}
    \end{proof}
\end{lemma}


\begin{lemma}
    \label{ParallelReductionProductAbstraction}
    \emph{The product and abstraction are preserved under parallel
    reduction}

    \begin{enumerate}
    \item
        $\ruleh{\lambda x^A. e \preduce c}
        {\exists B f. c = \lambda x^B. f
            \land A \preduce B \land e \preduce f}$

    \item
        $\ruleh{\Pi x^A. B \preduce c}
        {\exists C D. c = \Pi x^C. D
            \land A \preduce C \land B \preduce D}$
    \end{enumerate}

    \begin{proof} By induction on the premise. In both cases only one rule
    is syntactically possible which guarantees the existence of the
    corresponding terms.
    \end{proof}
\end{lemma}





\begin{theorem}
    \label{ParallelReductionDiamond}
    \emph{Parallel reduction is a diamond}
    \def\diap#1#2#3#4{\diamond {#1} {#2} {#3} {#4} {\beta_p} {\beta_b}}
    $$
    \diap a b c d
    $$

    \begin{proof} By induction on $a \preduce b$. Note that we keep the variable
        $c$ in the following universally quantified.

        \begin{enumerate}
        \item Reflexive
            $$
            \begin{array}{l|l}
                a \preduce a
                &
                \allbracket c {\diap a a c d}
            \end{array}
            $$
            To prove the goal on the right side we assume $a \preduce c$. Then
            we use $c$ for $d$ which satisfies the required property of $d$
            trivially.

        \item Redex
        $$
        \begin{array}{l|l}
            e \preduce f
            &
            \allbracket g {\diap e f g h}
            \\
            a \preduce b
            &
            \allbracket c {\diap a b c d}
            \\
            \hline
            (\lambda x^A. e) a \preduce f[x:=b]
            &
            \allbracket k {\diap {(\lambda x^A.e)a} {f[x:=b]} k n}
        \end{array}
        $$
        To prove the goal in the lower right corner we assume $(\lambda x^A.e)a
        \preduce k$ and do a case split on the construction of this relation.

            \begin{enumerate}
            \item Reflexive: In that case $k = (\lambda x^A. e) a$. We use $n =
            f[x:=b]$ which has the required property.


            \item Redex: In that case $k = g[x:=c]$ for some $g$ and $c$ with the
            properties $e \preduce g$ and $a \preduce c$.

            We have to find a term $n$ with $f[x:=b] \preduce n \land g[x:=c]
            \preduce n$. The term
            $$
                n = h[x:=d]
            $$
            with the terms $h$ and $d$ which exist by the induction hypotheses. It
            is easy to see that the properties
            $$
            \begin{array}{lll}
                f[x:=b] &\preduce& h[x:=d]
                \\
                g[x:=c] &\preduce& h[x:=d]
            \end{array}
            $$
            are satisfied because of the induction hypotheses and
            lemma~\ref{ParallelReductionSubstitution2}


            \item Product: This case is syntactically impossible because
            $(\lambda x^A.e) a$ cannot be a product.


            \item Abstraction: This case is syntactically impossible because
            $(\lambda x^A.e) a$ cannot be an abstraction.


            \item Application: In that case $k = (\lambda x^B. g) c$ for some
            $B$, $g$ and $c$ with $A \preduce B$, $e \preduce g$ and $a \preduce
            c$. Because of lemma~\ref{ParallelReductionProductAbstraction} we
            have chosen the more specific term $\lambda x^B. g$ instead of a
            more general term.

            We have to find a term $n$ which satisfies $(\lambda x^B. g) c
            \preduce n$ and $f[x:=b] \preduce n$.

            We use the term
            $$
                n = h[x:=d]
            $$
            with the terms $h$ and $d$ which exist by the induction hypotheses
            satisfying
            $$
            \begin{array}{lll}
                f &\preduce& h
                \\
                g &\preduce& h
                \\
                b &\preduce& d
                \\
                c &\preduce& d
            \end{array}
            $$
            Therefore the goals
            $$
            \begin{array}{lll}
                (\lambda x^B.g) c &\preduce& h[x:=d]
                \\
                f[x:=b] &\preduce& h[x:=d]
            \end{array}
            $$
            are satisfied
            \end{enumerate}


        \item Product
        $$
        \begin{array}{l|l}
            A \preduce C
            &
            \allbracket E {\diap A C E H}
            \\
            B \preduce D
            &
            \allbracket F {\diap B D F J}
            \\
            \hline
            \Pi x^A. B \preduce \Pi x^C. D
            &
            \allbracket
            {EF}
            {\diap {\Pi x^A.B} {\Pi x^C.D} {\Pi x^E.F} n}
        \end{array}
        $$
        In order to prove the goal in the lower right corner we assume $\Pi x^A.
        B \preduce \Pi x^E.F$.

        Because of lemma~\ref{ParallelReductionProductAbstraction} which says
        that parallel reduction preserves products we have chosen the more
        specific $\Pi x^E.F$ which satisfies $A \preduce E$ and $B \preduce F$
        instead of a more general term.

        From the induction hypotheses we conclude the existence of the terms $H$
        and $J$ such that
        $$
            n = \Pi x^H. J
        $$
        satisfies the required properties.

        \item Abstraction: Same reasoning as with product.

        \item Application
        $$
        \begin{array}{l|l}
            a \preduce c
            &
            \allbracket e {\diap a c e g}
            \\
            b \preduce d
            &
            \allbracket f {\diap b d f h}
            \\
            \hline
            a b \preduce c d
            &
            \allbracket k {\diap {ab} {cd} k n}
        \end{array}
        $$
        To prove the goal in the lower right corner we assume $a b \preduce k$
        and do a case split on the construction of this relation.

            \begin{enumerate}
            \item Reflexive: In that case $k = a b$. We use $n = c d$ which
            satisfies the required properties.

            \item Redex: In this case $a b$ has to be a redex, let's say
            $(\lambda x^A.m) b$. Therefore and because of
            lemma~\ref{ParallelReductionProductAbstraction} $a b \preduce c d$
            becomes $(\lambda x^A. m) b \preduce (\lambda x^B. o) d$ for some
            $B$ and $o$ with $A \preduce B$ and $m \preduce o$.

            $k$ has to be the reduct $p[x:=f]$ for some $p$, $f$ with $m
            \preduce p$ and $b \preduce f$.

            We have to find some term $n$ which satisfies
            $$
            \begin{array}{lll}
                (\lambda x^B. o) d &\preduce& n
                \\
                p[x:=f] &\preduce& n
            \end{array}
            $$

            From the first induction hypothesis and
            lemma~\ref{ParallelReductionProductAbstraction} we postulate the
            existence of $q$ with $o \preduce q$ and $p \preduce q$.

            From the second induction hypothesis we conclude the existence of
            some $h$ with with $d \preduce h$ and $f \preduce h$.

            Therefore the term
            $$
                n = q[x:=h]
            $$ satisfies the requirement.

            \item Product: This case is syntactically impossible because $a b$
            cannot be a product.

            \item Abstraction: This case is syntactically impossible because $a b$
            cannot be an abstraction.

            \item Application: In that case $k = e f$ for some terms $e$, $f$ which
            satisfy $a \preduce e$ and $b \preduce f$. We have to find some term
            $n$ which satisfies $c d \preduce n$ and $ e f \preduce n$.

            By the induction hypotheses there exist some terms $g$ and $h$
            satisfying $c \preduce g$, $e \preduce g$, $d \preduce h$ and $f
            \preduce h$ such that the term
            $$
                n = g h
            $$
            satisfies the requirement.
            \end{enumerate}
        \end{enumerate}
    \end{proof}
\end{theorem}



\begin{theorem}
    \label{ChurchRosser}
    \emph{Church Rosser theorem: Beta reduction is confluent.}

    \begin{proof}
        With the parallel beta reduction we have found a relation which is
        \begin{enumerate}
            \item a diamond (\ref{ParallelReductionDiamond})

            \item between beta reduction and its reflexive transitive closure
                (\ref{ReductionLeParallelReduction},
                \ref{ParallelReductionLeReductionStar})
        \end{enumerate}

        I.e. with parallel beta reduction we have found a diamond between beta
        reduction and its reflexive transitive closure which implies
        by~\ref{DiamondBetween} that beta reduction is confluent.
    \end{proof}
\end{theorem}



\subsection{Equivalent Terms}

One of the most important consequences of the Church Rosser theorem (i.e.
confluence of beta reduction) is the fact that beta equivelent terms have a
common reduct. I.e. for two beta equivalent terms $a$ and $b$ there exists
always a term $c$ to which both reduce
$$
    \begin{array}{lllll}
    a & & \betaeq & & b
    \\
    &\searrow_{\beta^*}
    &
    &\swarrow_{\beta^*}
    \\
    & & \exists c
    \end{array}
$$

\begin{theorem}
    \label{BetaEquivalentCommonReduct}
    \emph{Equivalent terms have a common reduct}.
    $$
    \ruleh {
        a \betaeq b
    }
    {
        \exists c. a \reducestar c \land b \reducestar c
    }
    $$

    \begin{proof}
        By induction on $a \betaeq b$.
        \begin{enumerate}
            \item Reflexive: In that case $a = b$. We use $c = a$ which satisfies
            the required properties trivially.

            \item Forward:
            $$
            \begin{array}{l|l}
                a \betaeq b
                &
                \exists d. a \reducestar d \land b \reducestar d
                \\
                b \reduce c
                \\
                \hline
                a \betaeq c
                &
                \exists e. a \reducestar e \land c \reducestar e
            \end{array}
            $$
            In order to prove the goal in the lower right corner we construct
            terms $d$ and $e$ which satisfy the following diagram and therefore
            $e$ satisfies the required properties.
            $$
            \begin{array}{lllll}
                a
                &\betaeq
                &b
                &\reduce
                &c
                \\
                &\searrow_{\beta^*}
                &\downarrow_{\beta^*}
                &
                &\downarrow_{\beta^*}
                \\
                &
                &d
                &\reducestar
                &e
            \end{array}
            $$
            $d$ exists by the induction hypothesis and $e$ exists by
            confluence~\ref{ChurchRosser}

            \item Backward:
            $$
            \begin{array}{l|l}
                a \betaeq b
                &
                \exists d. a \reducestar d \land b \reducestar d
                \\
                c \reduce b
                \\
                \hline
                a \betaeq c
                &
                \exists d. a \reducestar d \land c \reducestar d
            \end{array}
            $$
            In order to prove the goal in the lower right corner we construct
            term $d$ which satisfies the following diagram and therefore the
            required properties.
            $$
            \begin{array}{lllll}
                a
                &\betaeq
                &b
                &\fromrel{\beta}
                &c
                \\
                &\searrow_{\beta^*}
                &\downarrow_{\beta^*}
                &\swarrow_{\beta^*}
                \\
                &
                &d
            \end{array}
            $$
            The term $d$ exists by the induction hypothesis.
        \end{enumerate}
    \end{proof}
\end{theorem}




\subsection{Uniqueness of Normal Forms}

\begin{theorem}
    \label{NormalFormUnique}
    \emph{The normal form of a lambda term is unique}.
    $$
    \ruleh{
        t \reducestar u
        \\
        t \reducestar v
        \\
        \text{$u$ and $v$ are in normal form}
    }
    {
        u = v
    }
    $$

    \begin{proof}
        $u$ and $v$ are beta equivalent. According to the
        theorem~\ref{BetaEquivalentCommonReduct} both have to reduce to a common
        reduct, say $w$. Since both are in normal form, the common reduct has to
        be reached in zero reduction steps (no reduction step is possible in
        normal form) which is possible only if $u$ and $v$ are the same term.
    \end{proof}
\end{theorem}





\subsection{Equivalent Binders}

\begin{theorem}
    \label{EquivalentBinders}
    \emph{In beta equivalent binders i.e. terms of the form $\lambda x^A. e$ or
    $\Pi x^A.B$ corresponding subterms are beta equivalent}.
    $$
    \ruleh{
        \Pi x^A.B \betaeq \Pi x^C.D
    }
    {
        A \betaeq C \land B \betaeq D
    }
    \quad
    \ruleh{
        \lambda x^A.e \betaeq \lambda x^B.f
    }
    {
        A \betaeq B \land e \betaeq f
    }
    $$

    \begin{proof}
        We only prove the theorem for products (the proof for
        abstractions is practically the same).

        Since both products are beta equivalent they have
        by~\ref{BetaEquivalentCommonReduct} a common reduct.
        By~\ref{ReductionProductAbstraction} products (and abstractions) are
        preserved under beta reduction. Therefore the common reduct must have
        the form $\Pi x^E.F$ with $A \reducestar E$, $B \reducestar F$, $C
        \reducestar E$ and $D \reducestar F$.

        Since reduction implies beta equivalence and beta equivalence is
        transitive and symmetric we get $A \betaeq C$ and $B \betaeq D$.
    \end{proof}
\end{theorem}





\subsection{Binders are not Equivalent to Variables or Sorts}

\begin{theorem}
    \label{BinderNotEquivalentSortVariable}
    \emph{Binders (products or abstractions) cannot be beta equivalent to
    variables or sorts}.
    \begin{proof}
        We only prove
        $$
        \ruleh{
            \Pi x^A.B \betaeq s
        }
        {
            \perp
        }
        $$
        because the proofs of the other variants follow the same pattern.

        Assume $\Pi x^A.B \betaeq s$. Both terms must have a common reduct
        by~\ref{BetaEquivalentCommonReduct} which must be the sort $s$ since a
        sort is already in normal form. I.e. we get $\Pi x^A.B \reducestar s$.

        Since reduction preserves the form of binders
        by~\ref{ReductionProductAbstraction} $\Pi x^A.B \reducestar s$ is not
        possible and we get the desired contradiction.
    \end{proof}
\end{theorem}
