\documentclass[12pt]{article}
%\documentclass{article}

\usepackage{amsmath,amsthm,amssymb,xcolor,graphicx,tikz}
\usepackage{listings}
\usepackage{listings}
\usepackage{comment}
\usepackage{cite}
\usepackage[pdftex]{hyperref}
\usetikzlibrary{positioning}

\begin{document}

\def\BT{\text{BT}}
\def\NF{\text{NF}}

\def\torel#1{\stackrel{#1}\to}
\def\reduce{\torel{\beta}}
\def\reduceback{\stackrel{\beta}\leftarrow}
\def\reducestar{\torel{\beta^*}}
\def\reducelm{\torel{\beta_{\text{lm}}}}
\def\reducelmstar{\torel{\beta^*_{\text{lm}}}}
\def\reducehead{\torel{\beta_{\text{h}}}}
\def\betaeq{\stackrel{\beta}\sim}

\def\upfrom{\uparrow}



\def\vec#1{{\overrightarrow {#1}}}


\def\set#1{{\{#1\}}}

\def\betaeq{{\stackrel \beta \sim}}



\def\brackets#1{\left[ {#1} \right]}

\def\vertlist#1{
    \begin{array}{lllll}
    #1
    \end{array}
}

\def\bracklist#1{\brackets{ \vertlist{#1} }}


\def\cases#1{
    \left\{ \vertlist {#1} \right.
}


\def\rulev#1#2{
    \begin{array}{l}
        #1
        \\
        \hline
        #2
    \end{array}
}

\def\ruleh#1#2{
    \begin{array}{c}
        #1
        \\
        \hline
        #2
    \end{array}
}




% Theorems
\theoremstyle{definition}
\newtheorem{definition}{Definition}[section] % envname, text, number base
\newtheorem{theorem}[definition]{Theorem}    % envname, parent counter, text
\newtheorem{lemma}[definition]{Lemma}
\newtheorem{corollary}[definition]{Corollary}
\newtheorem{example}[definition]{Example}



\lstdefinelanguage{lam}
{ basewidth=0.45em,
  basicstyle=\small\tt,  % choices: small, footnotesize, scriptsize, tiny
  mathescape,
  %columns=flexible,
  keywords={
    let,
    where
  },
  keywordstyle=\color{blue},
  commentstyle=\color{brown},
  morecomment=[l]{--},
  morecomment=[n]{\{-}{-\}}
}

\lstdefinelanguage{ocaml}
{ basewidth=0.45em,
  basicstyle=\small\tt,  % choices: small, footnotesize, scriptsize, tiny
  mathescape,
  keywords={
    let,
    match,
    of,
    rec,
    type,
    with
  },
  keywordstyle=\color{blue},
  commentstyle=\color{brown},
}


\lstnewenvironment{lam}   {\lstset{language=lam}} {}
\lstnewenvironment{ocaml} {\lstset{language=ocaml}} {}



\title{
    Simulation of Lambda Terms in Lambda Calculus
}

\author{
    Helmut Brandl
    \\
    \scriptsize (firstname dot lastname at gmx dot net)
    \\
    \scriptsize Version x.x
}
\date{}

\maketitle

% \href {https://github.com/hbr/Lambda-Calculus} {Link text}
% \url {https://github.com/hbr/Lambda-Calculus}

\abstract{

    For comments, questions, error reporting feel free to open an issue at
    \url {https://github.com/hbr/Lambda-Calculus}
}







\tableofcontents






\section{Introduction}
%======================================================================








\section{Preliminaries}
%======================================================================

\cite{brandl-lambda-programming}

\cite{BARENDREGT1987191}







\section{Basics}
%======================================================================


\begin{lam}
    true  x y := x
    false y y := y

    and a b := a b false
    or  a b := a true b
    not a   := a false true
\end{lam}

\begin{lam}
    pair a b f := f a b
    fst  p     := p (\ a b := a)
    snd  p     := p (\ a b := b)
\end{lam}

\begin{lam}
    zero    f s  := s
    succ n  f s  := f (n f s) -- or n f (f s)

    one := succ zero
    two := succ one
    ...

    is-zero n :=
        n (\ x := false) true

    (+) a b := a succ b

    nat-rec n f s :=
        snd (n step (pair zero s)) where
          step p :=
            p (\ n0 r0 := pair (succ n0) (f n0 r0))


    pred n :=
        nat-rec (\ n0 r0 := n0) zero

    (-) a b := b pred a

    (<=) a b := is-zero (b - a)

    (=)  a b := (a <= b) and (b <= a)

    (<)  a b := (a <= b) and (not (b <= a))

    if-lt-eq-gt a b x y z
        -- if 'a < b' then return 'x',
        -- if 'a = b' then return 'y',
        -- if 'a > b' then return 'z'
    :=
        (a < b) x ((a = b) y z)
\end{lam}












\section{Simulator}
%======================================================================





\subsection{Encoding of Lambda Terms in Lambda Calculus}
%----------------------------------------------------------------------

We use the canonical form of lambda terms with De Bruijn indices which are
formed according to the grammar
$$
\vertlist{
    t &::=&
    i 
    & \text{De Bruijn index of the variable}
    \\
    &\mid&
    t t &
     \text{application}
    \\
    &\mid&
    \lambda t
    &\text{lambda abstraction}
}
$$

We need 3 constructors to encode an arbitrary canonical lambda term as a lambda
term. One constructor for variables (i.e. De Bruijn indices), one for
applications and one for lambda abstractions.

An encoded lambda term is a function with 4 variables:
\begin{enumerate}
    \item $b$: Church numeral representing the number of bound variables.

    \item $v$: Function transforming the Church numeral of the De Bruijn index
        of the variable and the number of bound variables into the result term
        of the variable.

    \item $a$: Funktion transforming the result terms of function term and the
        argument term into the result term of the application.

    \item $l$: Function transforming the result term of the body into the result
        term of the lambda abstraction.
\end{enumerate}

We assign types to the arguments of the encoded lambda term (note that the types
are just for our understanding of the arguments; the lambda calculus is untyped)
\begin{lam}
    b: Nat                  -- 'Nat' is a Church numeral

    v: Nat -> Nat -> R      -- 'R' is the type of the result

    a: R -> R

    l: R -> R -> R
\end{lam}
%
and an encoded lambda term has therefore the type
%
\begin{lam}
    Nat -> (Nat -> Nat -> R) -> (R -> R -> R) -> R
\end{lam}
%
%
Now it is easy to design the 3 necessary constructors.
\begin{lam}
    var k       b v a l  :=  v k b

    app x y     b v a l  :=  a (x b v a l) (y b v a l)

    lam x       b v a l  :=  l (x (succ b) v a l)
\end{lam}
%
\begin{enumerate}

    \item The constructor for a variable with De Bruijn index $k$ is a function
        mapping the 4 arguments into an application of the argument $v$ to the
        index $k$ and the number of the bound variables $b$ (note that numbers
        are encoded as Church numerals).

    \item The constructor for an application of the function term $x$ to the
        argument $y$ is a function mapping the 4 arguments into an application of
        the argument $a$ onto the results obtained from the function term $x b v
        a l$ and the argument term $y b v a l$.

    \item The constructor for a lambda abstraction with body $x$ is a function
        mapping the 4 arguments into an application of the argument $l$ onto the
        result obtained from the body applied to the arguments {x (succ b) v a
        l}. Note that the number of bound variables has to be increased by one
        because the lambda abstraction binds one more variable.
\end{enumerate}

We show for the lambda terms $\lambda x . x$ and $\lambda x y . x$ the
mathematical notation, the notation in program form, the mathematical notation
of the canonical form with De Bruijn indices and the encoded term in program
notation.

\begin{tabular}[t]{|l|l|l|l|}
    \hline
    Math
    & Progr Notation
    & De Bruijn
    & Encoded
    \\
    \hline
    $\lambda x. x$
    & {\tt $\backslash$ x := x}
    & $\lambda 0$
    & {\tt lam (var zero)}
    \\
    \hline
    $\lambda x y. x$
    & {\tt $\backslash$ x y := x}
    & $\lambda^2 1$
    & {\tt lam (lam (var one))}
    \\
    \hline
\end{tabular}

Note that the term in program notation and the encoded term in program notation
are different (to get the complete picture you have to expand the abbreviations
of {\tt lam}, {\tt var}, {\tt one} and {\tt zero}). I.e. the encoded term is
much more complex than the original term. The added complexity makes it possible
to inspect the structure of a lambda term in lambda calculus.


In order to demonstrate the handling of encoded lambda terms we show a function
which computes the number of subterms (including the term) of a lambda term.
The number of subterms is defined recursively. A variable has one subterm (the
variable itself), an application has one more subterm than the sum of the
subterms of the function and the argument term and an abstraction has one more
subterm than the number of subterms of the body.


\begin{lam}
    size t :=
        t
            zero
            (\ k b := one)
            (\ x y := succ (x + y))
            (\ x := succ x)
\end{lam}






\subsection{Recursion}
%----------------------------------------------------------------------



\section{Bibliography}
%======================================================================


\bibliography{IEEEabrv,../../bibliography}{}
\bibliographystyle{IEEEtran}


\end{document}
